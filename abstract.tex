\begin{abstract}
    
\section*{Abstract}
A common approach to studying motor control is defining motor synergies which can explain observed motor actions. This approach assumes that the large repertoire of observed movements is achieved by co-activating a small number of motor primitives in a specific spatiotemporal pattern. While this approach was successfully applied to describe movements, the neuronal manifestation of this organization is yet to be demonstrated. In the experiment, two macaque monkeys were trained to perform a two dimensional isometric wrist task to eight peripheral targets at two hand postures. Muscle activity from 11 forearm muscles was recorded by electromyography (EMG) while the monkeys performed the task. In a subset of recording-sessions single-pulse microstimulation in related sites of the motor cortex was applied while measuring the evoked muscle response. 

The results of this experiment supports the thesis of a neuronal manifestation of the synergies. They suggest that the profile of muscle activation during the task could be well explained by two to three muscle synergies which were extracted by multivariate analysis (NMF algorithm). These synergies were robust across recording sessions and hand postures. In addition the muscle responses evoked by cortical stimulation proved to be structured and could be explained by three muscle synergies. It was furthermore found that the synergies extracted from evoked responses show similarities with the synergies from natural movement. \todo{topographic organization?}

In summary, the work will present new evidence which indicates that the motor cortical representation of muscles is similar to the organization of naturally occurring movements with respect to a small number of synergies.
\end{abstract}