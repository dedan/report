%
%  untitled
%
%  Created by Stephan Gabler on 2011-01-09.
%  Copyright (c) 2011 __MyCompanyName__. All rights reserved.
%
\documentclass[]{article}

% Use utf-8 encoding for foreign characters
\usepackage[utf8]{inputenc}

% Setup for fullpage use
\usepackage{fullpage}

% Uncomment some of the following if you use the features
%
% Running Headers and footers
%\usepackage{fancyhdr}

% Multipart figures
%\usepackage{subfigure}

% More symbols
%\usepackage{amsmath}
%\usepackage{amssymb}
%\usepackage{latexsym}

% Surround parts of graphics with box
\usepackage{boxedminipage}

% Package for including code in the document
\usepackage{listings}

% If you want to generate a toc for each chapter (use with book)
\usepackage{minitoc}

% This is now the recommended way for checking for PDFLaTeX:
\usepackage{ifpdf}

%%%%%%%%%%%%%%%%%%%%%% TODO stuff

% graph painting
\usepackage{tikz}
% Command for inserting a todo item
\newcommand{\todo}[1]{%
% Add to todo list
\addcontentsline{tdo}{todo}{\protect{#1}}%
%
\begin{tikzpicture}[remember picture, baseline=-0.75ex]%
\node [coordinate] (inText) {};
\end{tikzpicture}%

% Make the margin par
\marginpar{%
\begin{tikzpicture}[remember picture]%
\definecolor{orange}{rgb}{1,0.5,0}
\draw node[draw=black, fill=orange, text width = 3cm] (inNote)
{#1};%
\end{tikzpicture}%
}%
%
\begin{tikzpicture}[remember picture, overlay]%
\draw[draw = orange, thick]
([yshift=-0.2cm] inText)
-| ([xshift=-0.2cm] inNote.west)
-| (inNote.west);
\end{tikzpicture}%
%
}%

\makeatletter \newcommand \listoftodos{\section*{Todo list} \@starttoc{tdo}}
 \newcommand\l@todo[2]
   {\par\noindent \textit{#2}, \parbox{10cm}{#1}\par} \makeatother


\title{Lab Rotation Report}
\author{Stephan Gabler\\
BCCN, Berlin}
%\email{stephan.gabler@gmail.com}

\date{\today}

\begin{document}

\ifpdf
\DeclareGraphicsExtensions{.pdf, .jpg, .tif}
\else
\DeclareGraphicsExtensions{.eps, .jpg}
\fi

\maketitle


\begin{abstract}
A common approach to studying motor control is defining motor synergies which can explain observed motor actions. This approach assumes that the large repertoire of observed movements is achieved by co-activating a small number of motor primitives in a specific spatiotemporal pattern. While this approach was successfully applied to describe movements, the neuronal manifestation of this organization is yet to be demonstrated. In the experiment, two macaque monkeys were trained to perform a two dimensional isometric wrist task to eight peripheral targets at two hand postures. Muscle activity from 11 forearm muscles was recorded by electromyography (EMG) while the monkeys performed the task. In a subset of recording-sessions single-pulse microstimulation in related sites of the motor cortex was applied while measuring the evoked muscle response. 

The results of this experiment supports the thesis of a neuronal manifestation of the synergies. They suggest that the profile of muscle activation during the task could be well explained by two to three muscle synergies which were extracted by multivariate analysis (NMF algorithm). These synergies were robust across recording sessions and hand postures. In addition the muscle responses evoked by cortical stimulation proved to be structured and could be explained by three muscle synergies. It was furthermore found that the synergies extracted from evoked responses show similarities with the synergies from natural movement. \todo{topographic organization?}

In summary, the work will present new evidence which indicates that the motor cortical representation of muscles is similar to the organization of naturally occurring movements with respect to a small number of synergies.
\end{abstract}



\section{Introduction}

%===============================================================================
% = about modular organization, dimensionality reduction and synergies in general =
%===============================================================================

The generation of adequate muscle activation patterns in order to achieve a certain goal is very complex. Difficulties arise from the high-dimensional and continuous sensory input space, from the dynamic and mostly non-linear transformations in between and the high-dimensional and continuous output space of the skeletomotor system. It was already shown in the famous study of Hubel and Wiesel~\cite{Hubel:1959p3833} that information in input space might be encoded by combinations of primitives, when neurons with receptive fields in forms of bars of several orientations were found. The newer theory of efficient coding argues similarly about the visual cortex and tries to find optimal sparse representations of sensory information which are governed by the statistical properties and regularities of the natural environment~\cite{Olshausen:1996p3611}.The formation of primitives governed by these regularities and modular organization might be able to simplify the difficulties that arise from the large number of degrees of freedom. Studies of the motor system, in particular of muscle activation patterns during movement, suggested the existence of fixed muscle activation patterns which form modular primitives in the output space to the skeletomotor system. 

By coupling of variables into a motor primitive (motor-\emph{synergy} in the following), the number of degrees of freedom can be dramatically reduced and therefore the modular organization into synergies might facilitate the generation of adequate muscle response patterns. Complex movement is then achieved by dynamic activation of linear combinations of these synergies.

Muscle synergies have been found in previous studies during natural movement and cutaneous stimulation of frogs~\cite{Tresch:1999p3783,Hart:2004p3786,Davella:2003p3784,Cheung:2005p3778}, cats~\cite{Ting:2004p3785} and humans~\cite{Merkle:1998p3780,Weiss:2004p3782,Krishnamoorthy:2003p3787,Olree:1995p3781,Ivanenko:2003p3779} by application of different methods of multivariate analysis. The activities during which synergies were found, range from postural responses to surface translation to completely unrestrained behavior in freely swimming and jumping frogs.

The research for this report is based on my BA-thesis and was done in a lab that studies the way motor command is translated into a detailed pattern of muscle activation within the corticospinal pathway. The working hypothesis of the lab is that the cortex does not control individual muscles; rather, it influences the activity of several functionally related groups of muscles. Furthermore, it is thought that detailed activation of muscles is generated by spinal neurons which integrate all the relevant dynamic parameters~\cite{yifat}.

Although modular organization into muscle synergies has been found in many cases, it is still not known what the underlying neural principles are. Following the working hypothesis of the lab, the neural correlate of synergies was assumed to be found in the motor cortex. Previous studies always investigated muscle synergies during natural movement. In the new approach of this study, muscle activation that was elicited by direct cortical stimulation was used for the search of muscle synergies.


\bigskip

The following questions will be addressed in this report:

\begin{enumerate}
	\item Can an underlying modular structure of muscle activation patterns be found during natural movement?
	\item If so, what is the neural basis of these synergies? Does the stimulation of single or small groups of neurons in the motor cortex activate synergies?
	\item If both found, what are the properties of these synergies and what is the relation between them?
	\item Is there a topographic organization with respect to muscle synergies in the motor cortex?
\end{enumerate}
	



\section{Methods} % (fold)
\label{sg:sec:methods}




% section methods (end)








\listoftodos

\bibliographystyle{plain}
\bibliography{bibdb}
\end{document}
