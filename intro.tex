\chapter{Introduction} % (fold)
\label{sg:cha:introduction}
%===============================================================================
% = about modular organization, dimensionality reduction and synergies in general =
%===============================================================================

The generation of adequate muscle activation patterns in order to achieve a certain goal is very complex. Difficulties arise from the high-dimensional and continuous sensory input space, from the dynamic and mostly non-linear transformations in between and the high-dimensional and continuous output space of the skeletomotor system. It was already shown in the famous study of Hubel and Wiesel \citet{Hubel:1959p3833} that information in input space might be encoded by combinations of primitives, when neurons with receptive fields in forms of bars of several orientations were found. The newer theory of efficient coding argues similarly about the visual cortex and tries to find optimal sparse representations of sensory information which are governed by the statistical properties and regularities of the natural environment~\citet{Olshausen:1996p3611}.The formation of primitives governed by these regularities and modular organization might be able to simplify the difficulties that arise from the large number of degrees of freedom. Studies of the motor system, in particular of muscle activation patterns during movement, suggested the existence of fixed muscle activation patterns which form modular primitives in the output space to the skeletomotor system. 

By coupling of variables into a motor primitive (motor-\emph{synergy} in the following), the number of degrees of freedom can be dramatically reduced and therefore the modular organization into synergies might facilitate the generation of adequate muscle response patterns. Complex movement is then achieved by dynamic activation of linear combinations of these synergies.

Muscle synergies have been found in previous studies during natural movement and cutaneous stimulation of frogs~\citet{Tresch:1999p3783,Hart:2004p3786,Davella:2003p3784,Cheung:2005p3778}, cats~\citet{Ting:2004p3785} and humans~\citet{Merkle:1998p3780,Weiss:2004p3782,Krishnamoorthy:2003p3787,Olree:1995p3781,Ivanenko:2003p3779} by application of different methods of multivariate analysis. The activities during which synergies were found, range from postural responses to surface translation to completely unrestrained behavior in freely swimming and jumping frogs.

The research for this report is based on my BA-thesis and was done in a lab that studies the way motor command is translated into a detailed pattern of muscle activation within the corticospinal pathway. The working hypothesis of the lab is that the cortex does not control individual muscles; rather, it influences the activity of several functionally related groups of muscles. Furthermore, it is thought that detailed activation of muscles is generated by spinal neurons which integrate all the relevant dynamic parameters~\citet{yifat}.

Although modular organization into muscle synergies has been found in many cases, it is still not known what the underlying neural principles are. Following the working hypothesis of the lab, the neural correlate of synergies was assumed to be found in the motor cortex. Previous studies always investigated muscle synergies during natural movement. In the new approach of this study, muscle activation that was elicited by direct cortical stimulation was used for the search of muscle synergies.


\bigskip

The following questions will be addressed in this report:

\begin{enumerate}
	\item Can an underlying modular structure of muscle activation patterns be found during natural movement?
	\item If so, what is the neural basis of these synergies? Does the stimulation of single or small groups of neurons in the motor cortex activate synergies?
	\item If both found, what are the properties of these synergies and what is the relation between them?
	\item Is there a topographic organization with respect to muscle synergies in the motor cortex?
\end{enumerate}

% section introduction (end)
